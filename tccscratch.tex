\documentclass[12pt, openright, a4paper, brazil, openany, oneside]{abntex2}

\usepackage{times}		
\usepackage[T1]{fontenc}	
\usepackage[utf8]{inputenc}	
\usepackage{indentfirst}	
\usepackage{color}			
\usepackage{graphicx}		
\usepackage{microtype} 		
\usepackage{multicol}
\usepackage{multirow}
\usepackage{lipsum}				
\usepackage[brazilian,hyperpageref]{backref}

\usepackage[small]{caption}

\usepackage[num,overcite]{abntex2cite}
\citebrackets()

\newtheorem{teo}{Teorema}
	 

\renewcommand{\backrefpagesname}{Citado na(s) página(s):~}
\renewcommand{\backref}{}
\renewcommand*{\backrefalt}[4]{
	\ifcase #1 %
		Nenhuma citação no texto.%
	\or
		Citado na página #2.%
	\else
		Citado #1 vezes nas páginas #2.%
	\fi}%

\titulo{Usando o Scratch no ensino da Matemática através da programação}
\autor{Sérgio Luís Soares Almeida \\ Matrícula 18/0006410}
\local{Brasília}
\data{2019}
\instituicao{%
  Universidade de Brasília -- UnB
  \par
  Departamento de Matemática
  \par
 PROFMAT}
\tipotrabalho{Estudo de Artigo}

\definecolor{black}{RGB}{0.0,0.0,0.0}


\makeatletter

\preambulo{Scratch no ensino de Matemática}

\hypersetup{pdftitle={\@title}, pdfauthor={\@author}, pdfsubject={\imprimirpreambulo}, pdfcreator={LaTeX with abnTeX2}, pdfkeywords={abnt}{latex}{abntex}{abntex2}{relatório técnico}, colorlinks=true, linkcolor=black, citecolor=black, filecolor=black, urlcolor=black, bookmarksdepth=4}
\makeatother

\setlength{\parindent}{1.3cm}


\setlength{\parskip}{0.2cm}  


\makeindex

\begin{document}






\selectlanguage{brazil}


\frenchspacing 
\imprimircapa
\imprimirfolhaderosto*
\ABNTEXchapterfont
\pdfbookmark[0]{\contentsname}{toc}
\tableofcontents*
\cleardoublepage
\textual

\chapter*[Introdução]{Introdução}
\addcontentsline{toc}{chapter}{Introdução}

Fazer introdução


\chapter{Pensamento Computacional}

O termo \textit{``computational thinking''} (pensamento computacional) passou a ser mais discutido a partir do artigo de Jeannette M. Wing em 2006, onde afirma que ``o pensamento computacional baseia-se no poder e nos limites dos processos de computação, sejam eles executados por um ser humano ou por uma máquina''\cite{wing} (Wing, 2006). Ainda, segundo Wing:

\begin{citacao}

``O pensamento computacional é uma habilidade fundamental para todos, não apenas para cientistas das computação. À leitura, escrita e aritmética, devemos acrescentar o pensamento computacional à capacidade analítica de cada criança'' (Wing, 2006)

\end{citacao}




No entanto o termo pensamento computacional não é definido em termos precisos. Em dois workshops sobre o âmbito e a natureza do pensamento computacional, patrocinados pela \textit{National Academy of Sciences} dos Estados Unidos da América, em 2009 e 2011, a \textit{National Research Council} publicou:

\begin{citacao}

``Os debates realizados no Workshop de fevereiro de 2009 não chegaram a um acordo geral entre os participantes sobre o conteúdo preciso de pensamento computacional, e muito menos a sua estrutura. No entanto, a falta de desacordo explícito sobre seus membros poderia ser entendida como refletindo uma intuitção compartilhada entre os participantes do workshop que o pensamento computacional, como um modo de pensamento, tem o seu próprio caráter distintivo'' \cite{NRC} (USA National Research Council, 2010)

\end{citacao}

As organizações \textit{International Society for Technology in Education} (ISTE) e a \textit{American Computer Science Teachers Association} (CSTA) trabalharam com pesquisadores da Ciência da Computação e das áreas de Humanas e construiram uma definição para o pensamento computacional pautada em elementos objetivos que pudesse nortear as atividades realizadas na Educação Básica identificando os seguintes conceitos: coleta de dados, análise de dados, representação de dados, decomposição do problema, abstração, algoritmos, automação, paralelização e simulação. Assim o pensamento computacional é definido como:

\begin{citacao}

``Pensamento computacional é um processo de solução de problemas que inclui (mas não está limitado a) as seguintes características: Formular problemas de uma forma que nos permita usar um computador e outras ferramentas para ajudar a resolvê-los; Organizar e analisar logicamente os dados; Representar dados através de abstrações, como modelos e simulações; Automatizar soluções por meio de pensamento algorítmico; Identificar, analisar e implementar possíveis soluções com o objetivo de alcançar a combinação mais eficientes e eficaz de etapas e recursos; Generalizar e transferir esse processo de solução de problemas para uma ampla variedade de problemas.''\cite{iste/csta}(ISTE/CSTA, 2011)

\end{citacao}

Observam ainda que essas habilidades são ``apoiadas e reforçadas por uma série de disposições ou atitudes que são dimensões essenciais do pensamento computacional'' assim como ``confiança em lidar com a complexidade, persistência em trabalhar com problemas difíceis, tolerância para a ambiguidade e capacidade de lidar com problemas abertos''\cite{iste/csta} (ISTE/CSTA, 2011).

A lógica de programação de computadores e de programação desplugada são atividades que integram muitas das características do pensamento computacional e muito tem sido feito para tornar a programação fácil de aprender.

\chapter*{Pensamento computacional e a matemática}

Fazer aqui ou no desenvolvimento do scratch?

\chapter{Objetos de Aprendizagem}

Os Objetos de Aprendizagem podem ser um grande aliado do professor como ferramenta de aprendizagem e de revisão de conceitos em vários conteúdos e propostas de ensino. A metodologia e o planejamento de suas aplicações são fatores determinantes para o sucesso da aprendizagem e para o desenvolvimento de estratégias para resolução de problemas, o seu uso deve estar previamente ligado aos objetivos do conteúdo e da aprendizagem que se quer alcançar.Podem ser de qualquer mídia ou formato, como uma apresentação em stop motion (animação quadro-a-quadro) ou uma simulação.

Não há um consenso sobre o conceito de Objetos de Aprendizagem. Segundo Aguiar/Flores ??? (2014) ``Sua definição surge de acordo com uma concepção própria dos autores acerca da utilidade e importância do Objeto para o ensino e a aprendizagem...'', desta forma, é importante que o conceito de um Objeto de Aprendizagem esteja de acordo com o objetivo do estudo e do planejamento do momento de aprendizagem. Entretanto, uma definição de Objetos de Aprendizagem significativa é dada por Tarouco (2003):


\begin{citacao}

``Um Objeto de Aprendizagem é qualquer recurso, suplementar ao processo de aprendizagem, que pode ser reusado para apoiar a aprendizagem, termo geralmente aplicado a materiais educacionais projetados e construidos em pequenos conjuntos visando a potencializar o processo de aprendizagem onde o recurso pode ser utilizado.''\cite{tarouco} (Tarouco, 2003)

\end{citacao}

Nesta definição devem ser destacadas as palavras ``recurso'', ``suplementar'' e ``reusado'', pois um Objeto de Aprendizagem não deve ser o objeto da aprendizagem e sim um recurso que o aluno possa utilizar para ajudá-lo a alcançar a aprendizagem do conteúdo previamente exposto e que esteja dentro dos objetivos pretendidos e que podem ser reutilizados não só para um conteúdo específico mas também em conjunto com outros Objetos de Aprendizagem podem ser combinados e recombinados dentro de um mesmo contexto e com conteúdos relacionados.

\chapter{Scratch na Educação}

 O Scratch é uma linguagem de programação feita por blocos lógicos capaz de integrar sons e imagens em histórias interativas, jogos e animações entre outros programas interativos desenvolvidos pelo usuário de forma intuitiva e sem o conhecimento prévio de outras linguagens de programação. Foi idealizado por Mitchel Resnick para pessoas com idades a partir de 8 anos e criado em 2007 pelo Lifelong Group, no MIT (Instituto de Tecnologia de Massachussets) Media Lab, onde continua sendo desenvolvido e moderado.  Hoje é usado em mais de 150 países e está disponível em mais de 40 idiomas.
 
 O Scratch é bastante acessível já que usa uma interface gráfica em que os programas são construidos através de blocos lógicos auto-encaixantes lembrando a montagem de um puzzle. A sua interface é de fácil compreensão (vide figura \ref{scr1}) e o uso de suas ferramentas muito intuitivas e sem complicações, possuindo três principais áreas:
 
\begin{enumerate}
\item Nesta área estão armazenados os blocos de comando, organizados a partir da sua funcionalidade.
\item A área de comandos a serem seguidos, onde os blocos serão encaixados
\item Palco ou simulador do Ecrã, onde se pode ver a simulação ou execução dos comandos.
\end{enumerate}

\begin{figure}[h]

    \center

    \includegraphics[width=12cm]{scratch1.png}
    \caption{interface do Scratch \label{scr1}}
    
\end{figure}

 Por ser uma linguagem de programação, toda ação deve ser dada na forma de comandos devidamente expressos nos seus blocos lógicos que são arrastados para uma área de comandos e conectados uns aos outros livremente (caso se encaixem). 
 
 As suas características de linguagem de programação e de um puzzle fazem com que o Scratch possa ser usado tanto como uma ferramenta para produção de Objetos de Aprendizagem como um meio de desenvolver o pensamento computacional do aluno.



\bibliography{bibliografia}

\end{document}