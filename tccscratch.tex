\documentclass[12pt, openright, a4paper, brazil, openany, oneside]{abntex2}

\usepackage{times}		
\usepackage[T1]{fontenc}	
\usepackage[utf8]{inputenc}	
\usepackage{indentfirst}	
\usepackage{color}			
\usepackage{graphicx}		
\usepackage{microtype} 		
\usepackage{multicol}
\usepackage{multirow}
\usepackage{lipsum}				
\usepackage[brazilian,hyperpageref]{backref}

\usepackage[num,overcite]{abntex2cite}
\citebrackets()

\newtheorem{teo}{Teorema}
	 

\renewcommand{\backrefpagesname}{Citado na(s) página(s):~}
\renewcommand{\backref}{}
\renewcommand*{\backrefalt}[4]{
	\ifcase #1 %
		Nenhuma citação no texto.%
	\or
		Citado na página #2.%
	\else
		Citado #1 vezes nas páginas #2.%
	\fi}%

\titulo{Usando o Scratch no ensino da Matemática através da programação}
\autor{Sérgio Luís Soares Almeida \\ Matrícula 18/0006410}
\local{Brasília}
\data{2019}
\instituicao{%
  Universidade de Brasília -- UnB
  \par
  Departamento de Matemática
  \par
 PROFMAT}
\tipotrabalho{Estudo de Artigo}

\definecolor{black}{RGB}{0.0,0.0,0.0}


\makeatletter

\preambulo{Scratch no ensino de Matemática}

\hypersetup{pdftitle={\@title}, pdfauthor={\@author}, pdfsubject={\imprimirpreambulo}, pdfcreator={LaTeX with abnTeX2}, pdfkeywords={abnt}{latex}{abntex}{abntex2}{relatório técnico}, colorlinks=true, linkcolor=black, citecolor=black, filecolor=black, urlcolor=black, bookmarksdepth=4}
\makeatother

\setlength{\parindent}{1.3cm}


\setlength{\parskip}{0.2cm}  


\makeindex

\begin{document}






\selectlanguage{brazil}


\frenchspacing 
\imprimircapa
\imprimirfolhaderosto*
\ABNTEXchapterfont
\pdfbookmark[0]{\contentsname}{toc}
\tableofcontents*
\cleardoublepage
\textual

\chapter*[Introdução]{Introdução}
\addcontentsline{toc}{chapter}{Introdução}

Fazer introdução


\chapter{Pensamento Computacional}

O termo \textit{``computational thinking''} (pensamento computacional) passou a ser mais discutido a partir do artigo de Jeannette M. Wing em 2006, onde afirma que ``o pensamento computacional baseia-se no poder e nos limites dos processos de computação, sejam eles executados por um ser humano ou por uma máquina''\cite{wing} (Wing, 2006). Ainda, segundo Wing:

\begin{citacao}

``O pensamento computacional é uma habilidade fundamental para todos, não apenas para cientistas das computação. À leitura, escrita e aritmética, devemos acrescentar o pensamento computacional à capacidade analítica de cada criança'' (Wing, 2006)

\end{citacao}




No entanto o termo pensamento computacional não é definido em termos precisos. Em dois workshops sobre o âmbito e a natureza do pensamento computacional, patrocinados pela \textit{National Academy of Sciences} dos Estados Unidos da América, em 2009 e 2011, a \textit{National Research Council} publicou:

\begin{citacao}

``Os debates realizados no Workshop de fevereiro de 2009 não chegaram a um acordo geral entre os participantes sobre o conteúdo preciso de pensamento computacional, e muito menos a sua estrutura. No entanto, a falta de desacordo explícito sobre seus membros poderia ser entendida como refletindo uma intuitção compartilhada entre os participantes do workshop que o pensamento computacional, como um modo de pensamento, tem o seu próprio caráter distintivo'' \cite{NRC} (USA National Research Council, 2010)

\end{citacao}

As organizações \textit{International Society for Technology in Education} (ISTE) e a \textit{American Computer Science Teachers Association} (CSTA) trabalharam com pesquisadores da Ciência da Computação e das áreas de Humanas e construiram uma definição para o pensamento computacional pautada em elementos objetivos que pudesse nortear as atividades realizadas na Educação Básica identificando os seguintes conceitos: coleta de dados, análise de dados, representação de dados, decomposição do problema, abstração, algoritmos, automação, paralelização e simulação. Assim o pensamento computacional é definido como:

\begin{citacao}

Pensamento computacional é um processo de solução de problemas que inclui (mas não está limitado a) as seguintes características: Formular problemas de uma forma que nos permita usar um computador e outras ferramentas para ajudar a resolvê-los; Organizar e analisar logicamente os dados; Representar dados através de abstrações, como modelos e simulações; Automatizar soluções por meio de pensamento algorítmico; Identificar, analisar e implementar possíveis soluções com o objetivo de alcançar a combinação mais eficientes e eficaz de etapas e recursos; Generalizar e transferir esse processo de solução de problemas para uma ampla variedade de problemas.\cite{iste/csta}(ISTE/CSTA, 2011)

\end{citacao}

Observam ainda que essas habilidades são "apoiadas e reforçadas por uma série de disposições ou atitudes que são dimensões essenciais do pensamento computacional" assim como "confiança em lidar com a complexidade, persistência em trabalhar com problemas difíceis, tolerância para a ambiguidade e capacidade de lidar com problemas abertos"\cite{iste/csta} (ISTE/CSTA, 2011).

A lógica de programação de computadores e de programação desplugada são atividades que integram muitas das características do pensamento computacional e muito tem sido feito para tornar a programação fácil de aprender.

\chapter*{Pensamento computacional e a matemática}

Fazer aqui ou no desenvolvimento do scratch?

\chapter{Objetos de Aprendizagem}



\bibliography{bibliografia}

\end{document}